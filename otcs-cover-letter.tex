\documentclass{letter}
\usepackage{hyperref}
\usepackage[a4paper, total={6in, 8in}]{geometry}
\usepackage{color}
\newcommand{\todo}[1]{\textcolor{red}{(to do: #1)}}
\signature{\mbox{Stephen McQuistin, Colin Perkins, and Marwan Fayed}}
\begin{document}

\begin{letter}{~}

\opening{Dear IFIP OTCS Editors:}

We would like to thank you for your invitation to submit to the inaugural
issue of the Open Transactions on Communication Systems journal. We are pleased
to have been able to submit an substantially extended version of our IFIP Networking 2016 paper,
``TCP Hollywood: An Unordered, Time-Lined, TCP for Networked Multimedia Applications''.
Additions and expansions are summarised in the following list.

\begin{itemize}

  \item \textbf{A full appendix devoted to reproducibility.} The appendix
  provides a justification for, as well as full description of, the evaluation
  environment. Instructions are included to reproduce our tests and
  measurements. TCP Hollywood sources have been made available online in
  support, as have custom scripts and tools to generate results.

  \item \textbf{Greatly expanded evaluations in Section V.} The conference draft
  necessarily focussed on the design and justification for TCP Hollywood; and
  associated analysis of expected benefits and behaviour. This came at the cost
  of detailed and rigorous evaluations, and is rectified in Section V. In this
  journal version we present results from our investigations so far. Four
  distinct measures are used to evaluate Hollywood against standard TCP, with
  observations recorded in Figures 7 through 16.

  \item \textbf{Additional background in Section II.} Here we have
  clarified the design goals, and used them as justification for a newly added
  subsection to describe service requirements (II.B.).

  \item \textbf{Edited the Related Works VII. section} to respond to reviewer comments regarding QUIC (see below); also to expand the discussion of TCP vs. UDP.

  \item \textbf{Introduction and Conclusion sections} have been edited to
  reflect the above changes.

\end{itemize}


%% =========================================

The reviewer comments responded to in this extended version of our IFIP Networking 2016
paper are outlined below, alongside a description of how this paper addresses them.

\textbf{Review 1: ``\emph{My main concern is that I am missing a critical analysis of the advantages
of TCP Hollywood given todays packet loss rates in networks.}''}

As noted in Section V.A. ("Simulator Setup"), the simulation of a fixed, random packet
loss rate is made difficult by the high variability of loss on the public Internet. The
evaluations added in the paper make use of three classes of typical cross-traffic,
resulting in a more realistic simulation.

\textbf{Review 2: ``\emph{TCP Hollywood seems like yet another
minimal improvement for TCP. QUIC, e.g., seems at least as likely to
improve jitter or loss based connectivity issues.}''}

%We view
TCP Hollywood and QUIC are complementary, rather than competing, protocols.
An updated Section VII (``Related Work") reflects this; it also discusses
the relative trade-off of using TCP vs UDP. UDP allows for implementations entirely
in user-space (greatly increasing deployability), whilst TCP maximises network reachability
(UDP may be blocked by enterprise firewalls).

Further, Section II ("Design Goals and API Requirements") has been updated to discuss the
broader architectural principles behind TCP Hollywood: we believe that transport-layer
evolution is made possible only by considering TCP and UDP as substrates. QUIC, in using
UDP as a substrate, is an example of this approach in practice.

\textbf{Review 3: ``\emph{the analysis provides a theoretic overview on how TCP Hollywood may
improve latency and head-of-line blocking through inconsistent
retransmission, but there is a simplified assumption "we assume broadly
symmetric network paths in this analysis." For general mobile networks,
this is not the case. As one way delay can be asymmetric in most mobile
access, this poses a concern for the usage of RTT/2 (in Formula 1)}''}

We agree that this is an issue with the analysis. The paper notes, in a footnote on page 5,
that this assumption does not hold in ADSL and cellular networks. Further analysis is needed,
but the broad conclusion of our analysis (i.e., that TCP Hollywood increases the usable
region of retransmissions) holds.

\textbf{Review 3: ``\emph{Rather weak evaluation for a system oriented paper}''}

Section V (``Performance Evaluations") expands on the analysis and deployability measurements
discussed in the IFIP Networking 2016 paper.

\textbf{Review 3: ``\emph{the source code is not publicly accessible, which makes it hard
to check the implementation to acquire more details.}''}

As noted in the Appendix (``Reproducibility"), the source code for the paper is publicly
accessible, including the kernel modifications and intermediary layer. The appendix also
gives detailed instructions on how this code fits together, and can be used to reproduce
the paper.

\textbf{Review 4: ``\emph{There is a separate buffer to hold per-message data. What is the impact of
the separate buffer on the CPU overhead and delay?}''}

While evaluations have not been carried out to determine the CPU impact of
TCP Hollywood, Section III (``Architecture \& Design") discusses the implementation
details (e.g., additional copies) that are likely to contribute to increased CPU overhead.
Based on these details, and anecdotal evidence from running the performance evaluations,
we do not believe that the CPU overhead of our modifications is significant.

Future work includes quantitive evaluations to measure CPU overhead and delay.

We are grateful to the reviewers for their comments. We are also grateful to the journal editors for the opportunity to further develop our work, as well as the criteria by which to improve and expand this work.

\closing{Best regards,}

\end{letter}
\end{document}
