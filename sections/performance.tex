% !TEX root = ../utltcp-paper.tex

In this section, we present initial performance evaluations. The primary goal 
is to show that the analysis presented in Section \ref{sec:analysis} holds,
while also showing that TCP Hollywood is useful for typical applications in realistic
network conditions.
We do this using the Mininet network emulator with a Linux implementation of TCP
Hollywood, simulating scenarios that represent typical networked multimedia
traffic from fixed-line residential networks.
This not only allows us to validate our analysis, but also can also form the
basis of future evaluation and deployment in the public Internet.


Section \ref{sec:simulator} describes the setup of the simulator used in the performance
evaluations, while Section \ref{sec:eval} discusses the parameters of the simulations,
and presents their results.

%--------------------------------------------------------------------------------------------------
\subsection{Simulator Setup}
\label{sec:simulator}

We use Mininet configured to emulate a dumbbell topology with a single bottleneck.
The bottleneck link of the dumbbell is an ADSL connection, with a download speed
of 8Mbps, and an upload speed of 1Mbps. ADSL is simulated because its use remains
widespread in large parts of the world, and is an environment where TCP Hollywood
is likely to show some benefit. The round-trip time is varied to allow exploration
of different behaviours, as will be discussed in Section \ref{sec:eval}.

We run TCP Hollywood flows representing the application under test, sharing the
bottleneck with cross-traffic. Three types of cross traffic are used:
single \emph{long-lived TCP} flows representing bulk downloads,
a bursty mix of short-lived TCP flows representing \emph{HTTP} traffic,
and multiple TCP connection representing an adaptive video download using
\emph{MPEG DASH}. Cross-traffic is used instead of a fixed random packet loss
rate; simulating such a loss rate is difficult, given the variability of
loss rates in the public Internet. Using different classes of typical cross
traffic is preferred.  We use the same parameters for generating cross
traffic for both protocols, but the simulations are non-deterministic so
the exact traffic pattern varies.

The Appendix further describes our choice of simulation parameters, the
rationale for our choice of simulation environment, and our approach to
ensuring reproducibility of results.

%--------------------------------------------------------------------------------------------------
\subsection{Evaluations}
\label{sec:eval}

In Section \ref{sec:analysis}, we showed how TCP and TCP Hollywood behaviour
splits into three regions, based on combinations of round-trip times and
play-out delays. In Figure \ref{diagram:inconsistentretrans}, these are
the green shaded
region, where TCP retransmissions are useful; the red shaded region, where TCP
retransmissions are not useful, but inconsistent retransmissions may be
beneficial; and finally, the unshaded region where no useful data can be
delivered -- the application is operating outside of its latency bounds.

\begin{figure}[t!]
	\includegraphics[width=\columnwidth]{figures/analysis-voip-inconsistent_region.pdf}
    \caption{Scenarios selected for VoIP application performance evaluations}
   	\label{fig:analysis-voip}
\end{figure}

Figure \ref{fig:analysis-voip} plots the same diagram, parameterised for the
Voice-over-IP application described in Section \ref{sec:analysis}, where
160-byte messages are sent every 20ms, to simulate the G.711 codec.
To validate the analysis, we simulate three scenarios using a round-trip time of
20ms and varying play-out delays:
(I) with play-out delay of
150ms, out-with the maximum delay tolerated by the application. At (II), the play-out delay
is 110ms; regular TCP and TCP Hollywood should provide the same timely goodput. At (III),
the play-out delay is 60ms, where TCP Hollywood should provide an increase in timely
goodput vs. standard TCP.
For each network scenario, and cross-traffic class, we plot:
\begin{itemize}
\item \emph{Timely good-put}:
                    TCP Hollywood is designed to improve performance by (i)
                    removing head-of-line blocking to reduce
                    latency; and (ii) introducing partial reliability to
                    increase utility.  Application-layer
                    latency (i.e., the time between the application sending
                    and receiving a message) gives insight into the
                    performance benefits, but fails to assess the trade-off
                    that TCP Hollywood makes between reliability and
                    timeliness.  Accordingly, we measure \emph{timely
                    good-put}, the rate at which useful bytes arrive at
                    the receiver. We define useful bytes as those that
                    arrive in time to be played out by the multimedia
                    application, with the correct timing to match the rest
                    of the media stream, and that have not already been
                    received.
\item \emph{Playout}: State of message playout for standard TCP and TCP Hollywood.
                    Messages that are successfully played out are shown in green; messages
                    that have expired (i.e., exceeded their maximum time bound) before their
                    play-out are shown in purple, and messages that arrived after their
                    play-out time are shown in orange. This plot begins when play-out starts,
                    removing the impact of the play-out buffer.
\item \emph{Latency}: Per-message latency for standard TCP and TCP Hollywood.
                    The latency of each message as it arrives at the application-layer is
                    shown. The sending application sends the message with a timestamp,
                    and the latency is determined at the receiver by comparing this to the
                    current time. Since simulations are performed on a
                    single host with Mininet, clocks are synchronised.
\item \emph{Segments}: Segment arrival for standard TCP and TCP Hollywood.
                    Segments that arrive in order are plotted in green. Segments that
                    arrive out-of-order are plotted in orange. Segments that fill an
                    earlier gap (i.e., retransmissions) are shown in purple. Segments that
                    were sent as inconsistent retransmissions are shown in blue, and are
                    additionally highlighted with marks on the border of the plot.
\end{itemize}

\begin{figure*}[t!]
\captionsetup[subfigure]{labelformat=empty}
\subfloat[Long-lived TCP]{\parbox{66mm}{
\begin{minipage}{5mm}
\rotatebox{90}{Timely goodput}
\end{minipage}
\begin{minipage}{61mm}
\includegraphics[width=61mm]{figures/perf-voip-adsl-uk-pd150ms-i-tg.pdf}\vspace{1mm}
\end{minipage}
\begin{minipage}{61mm}~\end{minipage}
}}
\subfloat[HTTP]{\parbox{61mm}{
\begin{minipage}{61mm}
\includegraphics[width=61mm]{figures/perf-voip-adsl-uk-pd150ms-ii-tg.pdf}\vspace{1mm}
\end{minipage}
\begin{minipage}{61mm}~\end{minipage}
}}
\subfloat[MPEG-DASH]{\parbox{61mm}{
\begin{minipage}{61mm}
\includegraphics[width=61mm]{figures/perf-voip-adsl-uk-pd150ms-iii-tg.pdf}\vspace{1mm}
\end{minipage}
\begin{minipage}{61mm}~\end{minipage}
}}
\caption{VoIP Scenario I: Timely goodput for standard TCP (Figure \ref{fig:voip-150-tcp}) and
                          TCP Hollywood (Figure \ref{fig:voip-150-tcph})}
\label{fig:voip-150-tg}
\end{figure*}

\begin{figure*}[t!]
\captionsetup[subfigure]{labelformat=empty}
\subfloat[Long-lived TCP]{\parbox{66mm}{
\begin{minipage}{5mm}
\rotatebox{90}{Playout}
\end{minipage}
\begin{minipage}{61mm}
\includegraphics[width=61mm]{figures/perf-voip-adsl-uk-pd150ms-i-tcp-playout.pdf}\vspace{1mm}
\end{minipage}
\begin{minipage}{5mm}
\rotatebox{90}{Message latency}
\end{minipage}
\begin{minipage}{61mm}
\includegraphics[width=61mm]{figures/perf-voip-adsl-uk-pd150ms-i-tcp-latency.pdf}\vspace{1mm}
\end{minipage}
\begin{minipage}{5mm}
\rotatebox{90}{Segments}
\end{minipage}
\begin{minipage}{61mm}
\includegraphics[width=61mm]{figures/perf-voip-adsl-uk-pd150ms-i-tcp-barcode.pdf}\vspace{1mm}
\end{minipage}
\begin{minipage}{61mm}~\end{minipage}
}}
\subfloat[HTTP]{\parbox{61mm}{
\begin{minipage}{61mm}
\includegraphics[width=61mm]{figures/perf-voip-adsl-uk-pd150ms-ii-tcp-playout.pdf}\vspace{1mm}
\end{minipage}
\begin{minipage}{61mm}
\includegraphics[width=61mm]{figures/perf-voip-adsl-uk-pd150ms-ii-tcp-latency.pdf}\vspace{1mm}
\end{minipage}
\begin{minipage}{61mm}
\includegraphics[width=61mm]{figures/perf-voip-adsl-uk-pd150ms-ii-tcp-barcode.pdf}\vspace{1mm}
\end{minipage}
\begin{minipage}{61mm}~\end{minipage}
}}
\subfloat[MPEG-DASH]{\parbox{61mm}{
\begin{minipage}{61mm}
\includegraphics[width=61mm]{figures/perf-voip-adsl-uk-pd150ms-iii-tcp-playout.pdf}\vspace{1mm}
\end{minipage}
\begin{minipage}{61mm}
\includegraphics[width=61mm]{figures/perf-voip-adsl-uk-pd150ms-iii-tcp-latency.pdf}\vspace{1mm}
\end{minipage}
\begin{minipage}{61mm}
\includegraphics[width=61mm]{figures/perf-voip-adsl-uk-pd150ms-iii-tcp-barcode.pdf}\vspace{1mm}
\end{minipage}
\begin{minipage}{61mm}~\end{minipage}
}}
\caption{VoIP Scenario I: Play-out buffer, message latencies, and segment arrival plots for standard TCP}
\label{fig:voip-150-tcp}
\end{figure*}

\begin{figure*}[t!]
\captionsetup[subfigure]{labelformat=empty}
\subfloat[Long-lived TCP]{\parbox{66mm}{
\begin{minipage}{5mm}
\rotatebox{90}{Playout}
\end{minipage}
\begin{minipage}{61mm}
\includegraphics[width=61mm]{figures/perf-voip-adsl-uk-pd150ms-i-tcph-playout.pdf}\vspace{1mm}
\end{minipage}
\begin{minipage}{5mm}
\rotatebox{90}{Message latency}
\end{minipage}
\begin{minipage}{61mm}
\includegraphics[width=61mm]{figures/perf-voip-adsl-uk-pd150ms-i-tcph-latency.pdf}\vspace{1mm}
\end{minipage}
\begin{minipage}{5mm}
\rotatebox{90}{Segments}
\end{minipage}
\begin{minipage}{61mm}
\includegraphics[width=61mm]{figures/perf-voip-adsl-uk-pd150ms-i-tcph-barcode.pdf}\vspace{1mm}
\end{minipage}
\begin{minipage}{61mm}~\end{minipage}
}}
\subfloat[HTTP]{\parbox{61mm}{
\begin{minipage}{61mm}
\includegraphics[width=61mm]{figures/perf-voip-adsl-uk-pd150ms-ii-tcph-playout.pdf}\vspace{1mm}
\end{minipage}
\begin{minipage}{61mm}
\includegraphics[width=61mm]{figures/perf-voip-adsl-uk-pd150ms-ii-tcph-latency.pdf}\vspace{1mm}
\end{minipage}
\begin{minipage}{61mm}
\includegraphics[width=61mm]{figures/perf-voip-adsl-uk-pd150ms-ii-tcph-barcode.pdf}\vspace{1mm}
\end{minipage}
\begin{minipage}{61mm}~\end{minipage}
}}
\subfloat[MPEG-DASH]{\parbox{61mm}{
\begin{minipage}{61mm}
\includegraphics[width=61mm]{figures/perf-voip-adsl-uk-pd150ms-iii-tcph-playout.pdf}\vspace{1mm}
\end{minipage}
\begin{minipage}{61mm}
\includegraphics[width=61mm]{figures/perf-voip-adsl-uk-pd150ms-iii-tcph-latency.pdf}\vspace{1mm}
\end{minipage}
\begin{minipage}{61mm}
\includegraphics[width=61mm]{figures/perf-voip-adsl-uk-pd150ms-iii-tcph-barcode.pdf}\vspace{1mm}
\end{minipage}
\begin{minipage}{61mm}~\end{minipage}
}}
\caption{VoIP Scenario I: Play-out buffer, message latencies, and segment arrival plots for TCP Hollywood}
\label{fig:voip-150-tcph}
\end{figure*}

The simulation results for Scenario I are shown in Figures \ref{fig:voip-150-tg},
\ref{fig:voip-150-tcp}, and \ref{fig:voip-150-tcph}.
There are three groups of results:
Figure \ref{fig:voip-150-tg} shows timely goodput for both standard TCP and
for TCP Hollywood competing with each of the three types of cross traffic.
Figure \ref{fig:voip-150-tcp} shows, for standard TCP, and with each type
of cross traffic, whether segments were played out on time (top), the per
message latency as measured at the application layer (middle), and whether
each segment was received, retransmitted, or inconsistently retransmitted
(bottomw).
Finally, Figure \ref{fig:voip-150-tcph}, plots the same information, but
for TCP Hollywood.

\begin{figure*}[t!]
\captionsetup[subfigure]{labelformat=empty}
\subfloat[Long-lived TCP]{\parbox{66mm}{
\begin{minipage}{5mm}
\rotatebox{90}{Timely goodput}
\end{minipage}
\begin{minipage}{61mm}
\includegraphics[width=61mm]{figures/perf-voip-adsl-uk-pd110ms-i-tg.pdf}\vspace{1mm}
\end{minipage}
\begin{minipage}{61mm}~\end{minipage}
}}
\subfloat[HTTP]{\parbox{61mm}{
\begin{minipage}{61mm}
\includegraphics[width=61mm]{figures/perf-voip-adsl-uk-pd110ms-ii-tg.pdf}\vspace{1mm}
\end{minipage}
\begin{minipage}{61mm}~\end{minipage}
}}
\subfloat[MPEG-DASH]{\parbox{61mm}{
\begin{minipage}{61mm}
\includegraphics[width=61mm]{figures/perf-voip-adsl-uk-pd110ms-iii-tg.pdf}\vspace{1mm}
\end{minipage}
\begin{minipage}{61mm}~\end{minipage}
}}
\caption{VoIP Scenario II: Timely goodput for standard TCP (Figure \ref{fig:voip-110-tcp}) and
                          TCP Hollywood (Figure \ref{fig:voip-110-tcph})}
\label{fig:voip-110-tg}
\end{figure*}

\begin{figure*}[t!]
\captionsetup[subfigure]{labelformat=empty}
\subfloat[Long-lived TCP]{\parbox{66mm}{
\begin{minipage}{5mm}
\rotatebox{90}{Playout}
\end{minipage}
\begin{minipage}{61mm}
\includegraphics[width=61mm]{figures/perf-voip-adsl-uk-pd110ms-i-tcp-playout.pdf}\vspace{1mm}
\end{minipage}
\begin{minipage}{5mm}
\rotatebox{90}{Message latency}
\end{minipage}
\begin{minipage}{61mm}
\includegraphics[width=61mm]{figures/perf-voip-adsl-uk-pd110ms-i-tcp-latency.pdf}\vspace{1mm}
\end{minipage}
\begin{minipage}{5mm}
\rotatebox{90}{Segments}
\end{minipage}
\begin{minipage}{61mm}
\includegraphics[width=61mm]{figures/perf-voip-adsl-uk-pd110ms-i-tcp-barcode.pdf}\vspace{1mm}
\end{minipage}
\begin{minipage}{61mm}~\end{minipage}
}}
\subfloat[HTTP]{\parbox{61mm}{
\begin{minipage}{61mm}
\includegraphics[width=61mm]{figures/perf-voip-adsl-uk-pd110ms-ii-tcp-playout.pdf}\vspace{1mm}
\end{minipage}
\begin{minipage}{61mm}
\includegraphics[width=61mm]{figures/perf-voip-adsl-uk-pd110ms-ii-tcp-latency.pdf}\vspace{1mm}
\end{minipage}
\begin{minipage}{61mm}
\includegraphics[width=61mm]{figures/perf-voip-adsl-uk-pd110ms-ii-tcp-barcode.pdf}\vspace{1mm}
\end{minipage}
\begin{minipage}{61mm}~\end{minipage}
}}
\subfloat[MPEG-DASH]{\parbox{61mm}{
\begin{minipage}{61mm}
\includegraphics[width=61mm]{figures/perf-voip-adsl-uk-pd110ms-iii-tcp-playout.pdf}\vspace{1mm}
\end{minipage}
\begin{minipage}{61mm}
\includegraphics[width=61mm]{figures/perf-voip-adsl-uk-pd110ms-iii-tcp-latency.pdf}\vspace{1mm}
\end{minipage}
\begin{minipage}{61mm}
\includegraphics[width=61mm]{figures/perf-voip-adsl-uk-pd110ms-iii-tcp-barcode.pdf}\vspace{1mm}
\end{minipage}
\begin{minipage}{61mm}~\end{minipage}
}}
\caption{VoIP Scenario II: Play-out buffer, message latencies, and segment arrival plots for standard TCP}
\label{fig:voip-110-tcp}
\end{figure*}

\begin{figure*}[t!]
\captionsetup[subfigure]{labelformat=empty}
\subfloat[Long-lived TCP]{\parbox{66mm}{
\begin{minipage}{5mm}
\rotatebox{90}{Playout}
\end{minipage}
\begin{minipage}{61mm}
\includegraphics[width=61mm]{figures/perf-voip-adsl-uk-pd110ms-i-tcph-playout.pdf}\vspace{1mm}
\end{minipage}
\begin{minipage}{5mm}
\rotatebox{90}{Message latency}
\end{minipage}
\begin{minipage}{61mm}
\includegraphics[width=61mm]{figures/perf-voip-adsl-uk-pd110ms-i-tcph-latency.pdf}\vspace{1mm}
\end{minipage}
\begin{minipage}{5mm}
\rotatebox{90}{Segments}
\end{minipage}
\begin{minipage}{61mm}
\includegraphics[width=61mm]{figures/perf-voip-adsl-uk-pd110ms-i-tcph-barcode.pdf}\vspace{1mm}
\end{minipage}
\begin{minipage}{61mm}~\end{minipage}
}}
\subfloat[HTTP]{\parbox{61mm}{
\begin{minipage}{61mm}
\includegraphics[width=61mm]{figures/perf-voip-adsl-uk-pd110ms-ii-tcph-playout.pdf}\vspace{1mm}
\end{minipage}
\begin{minipage}{61mm}
\includegraphics[width=61mm]{figures/perf-voip-adsl-uk-pd110ms-ii-tcph-latency.pdf}\vspace{1mm}
\end{minipage}
\begin{minipage}{61mm}
\includegraphics[width=61mm]{figures/perf-voip-adsl-uk-pd110ms-ii-tcph-barcode.pdf}\vspace{1mm}
\end{minipage}
\begin{minipage}{61mm}~\end{minipage}
}}
\subfloat[MPEG-DASH]{\parbox{61mm}{
\begin{minipage}{61mm}
\includegraphics[width=61mm]{figures/perf-voip-adsl-uk-pd110ms-iii-tcph-playout.pdf}\vspace{1mm}
\end{minipage}
\begin{minipage}{61mm}
\includegraphics[width=61mm]{figures/perf-voip-adsl-uk-pd110ms-iii-tcph-latency.pdf}\vspace{1mm}
\end{minipage}
\begin{minipage}{61mm}
\includegraphics[width=61mm]{figures/perf-voip-adsl-uk-pd110ms-iii-tcph-barcode.pdf}\vspace{1mm}
\end{minipage}
\begin{minipage}{61mm}~\end{minipage}
}}
\caption{VoIP Scenario II: Play-out buffer, message latencies, and segment arrival plots for TCP Hollywood}
\label{fig:voip-110-tcph}
\end{figure*}


% ================


\begin{figure*}[t!]
\captionsetup[subfigure]{labelformat=empty}
\subfloat[Long-lived TCP]{\parbox{66mm}{
\begin{minipage}{5mm}
\rotatebox{90}{Timely goodput}
\end{minipage}
\begin{minipage}{61mm}
\includegraphics[width=61mm]{figures/perf-voip-adsl-uk-pd60ms-i-tg.pdf}\vspace{1mm}
\end{minipage}
\begin{minipage}{61mm}~\end{minipage}
}}
\subfloat[HTTP]{\parbox{61mm}{
\begin{minipage}{61mm}
\includegraphics[width=61mm]{figures/perf-voip-adsl-uk-pd60ms-ii-tg.pdf}\vspace{1mm}
\end{minipage}
\begin{minipage}{61mm}~\end{minipage}
}}
\subfloat[MPEG-DASH]{\parbox{61mm}{
\begin{minipage}{61mm}
\includegraphics[width=61mm]{figures/perf-voip-adsl-uk-pd60ms-iii-tg.pdf}\vspace{1mm}
\end{minipage}
\begin{minipage}{61mm}~\end{minipage}
}}
\caption{VoIP Scenario III: Timely goodput for standard TCP (Figure \ref{fig:voip-60-tcp}) and
                          TCP Hollywood (Figure \ref{fig:voip-60-tcph})}
\label{fig:voip-60-tg}
\end{figure*}

\begin{figure*}[t!]
\captionsetup[subfigure]{labelformat=empty}
\subfloat[Long-lived TCP]{\parbox{66mm}{
\begin{minipage}{5mm}
\rotatebox{90}{Playout}
\end{minipage}
\begin{minipage}{61mm}
\includegraphics[width=61mm]{figures/perf-voip-adsl-uk-pd60ms-i-tcp-playout.pdf}\vspace{1mm}
\end{minipage}
\begin{minipage}{5mm}
\rotatebox{90}{Message latency}
\end{minipage}
\begin{minipage}{61mm}
\includegraphics[width=61mm]{figures/perf-voip-adsl-uk-pd60ms-i-tcp-latency.pdf}\vspace{1mm}
\end{minipage}
\begin{minipage}{5mm}
\rotatebox{90}{Segments}
\end{minipage}
\begin{minipage}{61mm}
\includegraphics[width=61mm]{figures/perf-voip-adsl-uk-pd60ms-i-tcp-barcode.pdf}\vspace{1mm}
\end{minipage}
\begin{minipage}{61mm}~\end{minipage}
}}
\subfloat[HTTP]{\parbox{61mm}{
\begin{minipage}{61mm}
\includegraphics[width=61mm]{figures/perf-voip-adsl-uk-pd60ms-ii-tcp-playout.pdf}\vspace{1mm}
\end{minipage}
\begin{minipage}{61mm}
\includegraphics[width=61mm]{figures/perf-voip-adsl-uk-pd60ms-ii-tcp-latency.pdf}\vspace{1mm}
\end{minipage}
\begin{minipage}{61mm}
\includegraphics[width=61mm]{figures/perf-voip-adsl-uk-pd60ms-ii-tcp-barcode.pdf}\vspace{1mm}
\end{minipage}
\begin{minipage}{61mm}~\end{minipage}
}}
\subfloat[MPEG-DASH]{\parbox{61mm}{
\begin{minipage}{61mm}
\includegraphics[width=61mm]{figures/perf-voip-adsl-uk-pd60ms-iii-tcp-playout.pdf}\vspace{1mm}
\end{minipage}
\begin{minipage}{61mm}
\includegraphics[width=61mm]{figures/perf-voip-adsl-uk-pd60ms-iii-tcp-latency.pdf}\vspace{1mm}
\end{minipage}
\begin{minipage}{61mm}
\includegraphics[width=61mm]{figures/perf-voip-adsl-uk-pd60ms-iii-tcp-barcode.pdf}\vspace{1mm}
\end{minipage}
\begin{minipage}{61mm}~\end{minipage}
}}
\caption{VoIP Scenario III: Play-out buffer, message latencies, and segment arrival plots for standard TCP}
\label{fig:voip-60-tcp}
\end{figure*}

\begin{figure*}[t!]
\captionsetup[subfigure]{labelformat=empty}
\subfloat[Long-lived TCP]{\parbox{66mm}{
\begin{minipage}{5mm}
\rotatebox{90}{Playout}
\end{minipage}
\begin{minipage}{61mm}
\includegraphics[width=61mm]{figures/perf-voip-adsl-uk-pd60ms-i-tcph-playout.pdf}\vspace{1mm}
\end{minipage}
\begin{minipage}{5mm}
\rotatebox{90}{Message latency}
\end{minipage}
\begin{minipage}{61mm}
\includegraphics[width=61mm]{figures/perf-voip-adsl-uk-pd60ms-i-tcph-latency.pdf}\vspace{1mm}
\end{minipage}
\begin{minipage}{5mm}
\rotatebox{90}{Segments}
\end{minipage}
\begin{minipage}{61mm}
\includegraphics[width=61mm]{figures/perf-voip-adsl-uk-pd60ms-i-tcph-barcode.pdf}\vspace{1mm}
\end{minipage}
\begin{minipage}{61mm}~\end{minipage}
}}
\subfloat[HTTP]{\parbox{61mm}{
\begin{minipage}{61mm}
\includegraphics[width=61mm]{figures/perf-voip-adsl-uk-pd60ms-ii-tcph-playout.pdf}\vspace{1mm}
\end{minipage}
\begin{minipage}{61mm}
\includegraphics[width=61mm]{figures/perf-voip-adsl-uk-pd60ms-ii-tcph-latency.pdf}\vspace{1mm}
\end{minipage}
\begin{minipage}{61mm}
\includegraphics[width=61mm]{figures/perf-voip-adsl-uk-pd60ms-ii-tcph-barcode.pdf}\vspace{1mm}
\end{minipage}
\begin{minipage}{61mm}~\end{minipage}
}}
\subfloat[MPEG-DASH]{\parbox{61mm}{
\begin{minipage}{61mm}
\includegraphics[width=61mm]{figures/perf-voip-adsl-uk-pd60ms-iii-tcph-playout.pdf}\vspace{1mm}
\end{minipage}
\begin{minipage}{61mm}
\includegraphics[width=61mm]{figures/perf-voip-adsl-uk-pd60ms-iii-tcph-latency.pdf}\vspace{1mm}
\end{minipage}
\begin{minipage}{61mm}
\includegraphics[width=61mm]{figures/perf-voip-adsl-uk-pd60ms-iii-tcph-barcode.pdf}\vspace{1mm}
\end{minipage}
\begin{minipage}{61mm}~\end{minipage}
}}
\caption{VoIP Scenario III: Play-out buffer, message latencies, and segment arrival plots for TCP Hollywood}
\label{fig:voip-60-tcph}
\end{figure*}

In scenario I, the combination of a 150ms play-out delay and 10ms network
delay (20ms RTT) is chosen such that the application is operating beyond
its maximum one-way delay bound. That is, packets cannot arrive in time to
meet the deadline, irrespective of whether standard TCP or TCP Hollywood
are used. This is clear in the figures: the timely goodput (Figure
\ref{fig:voip-150-tg}) is zero for both protocols, and the playout plots
show all messages have expired.  Both standard TCP and TCP Hollywood behave
as expected: playout in this scenario is not feasible for either protocol.

The results for Scenario II are shown in Figures \ref{fig:voip-110-tg},
\ref{fig:voip-110-tcp}, and \ref{fig:voip-110-tcph}.
Figure \ref{fig:voip-110-tg} plots the timely goodput for both standard TCP
and TCP Hollywood, while Figures \ref{fig:voip-110-tcp} and \ref{fig:voip-110-tcph} 
show the playout of the segments, application-layer message latency, and
segment arrivals for the two protocols. Each figure shows performance with
each of the three types of cross traffic. 

In this scenario, the playout delay is set to 110ms, chosen such that the
combination of playout delay and network round-trip time allows
retransmissions to arrive before the lost and retransmitted segment is 
due to be played out. Accordingly, the performance of standard TCP and 
TCP Hollywood should be comparable. We see this in the figures. While
there is variation in the timely goodput (Figure \ref{fig:voip-110-tg}) 
due to variations in the cross traffic, there is no significant difference
between standard TCP and TCP Hollywood. The other metrics (Figures 
\ref{fig:voip-110-tcp} and \ref{fig:voip-110-tcph}) are also comparable,
although since the pattern of segment arrivals varies depending on the
cross traffic, we see differences in instantaneous message latency and
playout.


\begin{figure*}[t!]
\captionsetup[subfigure]{labelformat=empty}
\subfloat[\hspace{5mm}(i) Standard TCP]{\parbox{96mm}{
\begin{minipage}{5mm}
\rotatebox{90}{Playout}
\end{minipage}
\begin{minipage}{91mm}
\includegraphics[width=91mm]{figures/perf-playout-voip-tcp-zoom.pdf}\vspace{1mm}
\end{minipage}
\begin{minipage}{5mm}
\rotatebox{90}{Message latency}
\end{minipage}
\begin{minipage}{91mm}
\includegraphics[width=91mm]{figures/perf-latency-voip-tcp-zoom.pdf}\vspace{1mm}
\end{minipage}
\begin{minipage}{5mm}
\rotatebox{90}{Segments}
\end{minipage}
\begin{minipage}{91mm}
\includegraphics[width=91mm]{figures/perf-barcode-voip-tcp-zoom.pdf}\vspace{1mm}
\end{minipage}
\begin{minipage}{91mm}~\end{minipage}
}}
\subfloat[(ii) TCP Hollywood]{\parbox{91mm}{
\begin{minipage}{91mm}
\includegraphics[width=91mm]{figures/perf-playout-voip-tcph-zoom.pdf}\vspace{1mm}
\end{minipage}
\begin{minipage}{91mm}
\includegraphics[width=91mm]{figures/perf-latency-voip-tcph-zoom.pdf}\vspace{1mm}
\end{minipage}
\begin{minipage}{91mm}
\includegraphics[width=91mm]{figures/perf-barcode-voip-tcph-zoom.pdf}\vspace{1mm}
\end{minipage}
\begin{minipage}{91mm}~\end{minipage}
}}
\caption{VoIP: comparing the behaviour of standard TCP and TCP Hollywood when a segment is lost}
\label{fig:voip-comparison}
\end{figure*}

Figure \ref{fig:voip-110-tcph} also demonstrates a sensitivity of TCP
Hollywood to the accuracy of the TCP RTT estimate. The segment plots
show a significant number of inconsistent retransmissions, marked by
ticks on the outside of the plots. These are strongly correlated with
latency spikes, and occur when short-term increases in delay push the
delivery time estimate for the packets past their deadline, causing
TCP Hollywood to send inconsistent retransmissions.
We believe the spikes in latency due to queuing are causing the playout
time of the packets to oscillate around their deadline. Sometimes this
hurts TCP Hollywood, since inconsistent retransmissions are incorrectly
sent; equally, though, it hurts standard TCP since there are times where
inconsistent retransmissions are needed.  There is no significant impact 
on the timely goodput.

Finally, the results for Scenario III are shown in Figures
\ref{fig:voip-60-tg}, \ref{fig:voip-60-tcp}, and \ref{fig:voip-60-tcph}.
This is the primary scenario of interest: the combination of playout delay
and network round trip time is such that standard retransmissions will not
arrive in time to be played out, causing head-of-line blocking that will
affect performance with standard TCP. In contrast, the use of inconsistent
retransmissions and an alternate, message based, API that avoids head of
line blocking should allow TCP Hollywood to perform well.

Considering the timely goodput, Figure \ref{fig:voip-60-tg}, we see that
both protocols are comparable when network conditions
are good. However, with long-lived TCP cross traffic and MPEG DASH cross
traffic, there are several regions where the timely goodput of standard
TCP drops sharply below that of TCP Hollywood. It can readily be seen 
that these correlate with the spikes in Message Latency seen in Figure
\ref{fig:voip-60-tcp}. The corresponding spikes in latency for the TCP
Hollywood traces, Figure \ref{fig:voip-60-tcph}, do not show the drop
in timely goodput, but do correlate with inconsistent retransmissions.
This is the behaviour we expect: standard TCP flows stall due to head
of line blocking with waiting for retransmissions that arrive too late
to be played out, while TCP Hollywood sends an inconsistent retransmission
that arrives in time to be useful. The TCP Hollywood receiver sees a burst
of packet loss, while the standard TCP receiver sees no loss, but a longer
burst of packets being delivered late. TCP Hollywood achieves better
\emph{timely} goodput, even though the amount of packet loss at the
application is higher. 

The behaviour of standard TCP and TCP Hollywood in the case of segment loss
is explored in more detail in Figure \ref{fig:voip-comparison}.  The loss
events are drawn from the Scenario III long-lived TCP plots, at around 108
seconds for standard TCP and 84 seconds for TCP Hollywood.  
Impact of removing head-of-line blocking is clear: fewer of the messages
preceding the delivery of a retransmission miss their deadline.
% FIXME: this should be expanded

% \todo{Discuss inconsistent retransmissions once plotted}
% \todo{Has it been checked that these are actual inconsistent retransmissions yet?}
% 
% plot the timely goodput, and playout, latency, and barcode plots for TCP
% and TCP Hollywood for Scenario III. On average, TCP Hollywood achieves more
% timely goodput than standard TCP.  \todo{expand this}
% 
% \todo{Add numbers to support this.}
% 
% \begin{table*}[h]
%  \centering
%  \normalsize
%   \input{figures/perfeval-results.tex}
%  \vspace{4mm}
%   \caption{Performance evaluation results}
% \label{tab:perfeval-results}
% \end{table*}
% FIXME: This table is not especially meaningful, since it's missing
% confidence intervals. Need to re-run the simulations enough to get
% meaningful error bar statistics. 


In summary, this performance evaluation shows the validity of the
analysis presented in Section \ref{sec:analysis}. TCP Hollywood is shown
to improve timely goodput in cases where standard TCP retransmissions would
arrive too late to be useful, converting a long duration playout stall into 
a shorter packet loss event.

