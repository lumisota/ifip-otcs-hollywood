% !TEX root = ../utltcp-paper.tex

% The design of TCP Hollywood makes only one wire-visible modification to TCP:
% inconsistent retransmissions. Middleboxes along the path may see multiple
%segments with the same TCP sequence number, but with different payloads.

% We investigate the feasibility of deploying TCP Hollywood, using results
% from initial experiments with a FreeBSD implementation on
% residential and mobile networks in the UK.
%
% \todo{Remove reference to FreeBSD? Confuses the point?}

In an ossified Internet, TCP Hollywood must survive middlebox behaviours if it
is to be deployable. In this section we present positive results from a
preliminary investigation of TCP Hollyood between a residential and multiple
mobile networks in the UK.

TCP Hollywood is designed to be entirely compatible with TCP. The only
on-the-wire visible difference between a TCP Hollywood flow and a standard TCP
flow appears within the payload data carried by inconsistent retransmissions.
Recall from Section~\ref{subsec:sender} that inconsistent retransmissions carry
new payload data inside of segments with previously transmitted sequence
numbers. This modification is invisible to receivers and middleboxes that only
process TCP/IP headers, but is visible to middleboxes that use deep packet
inspection if they compare the contents of a retransmitted packet with the
original data. Depending on the configuration such behaviour may disrupt the
connection. For example, a firewall may interpret inconsistent retransmissions
as belonging to a man-on-the-side attack, and reset the connection.

We conducted experiments with a live deployment of TCP Hollywood to obtain an
initial assessment on whether such middleboxes exist, and what impact they have.
A TCP Hollywood server was setup on a public IP address. The server was
configured to always send inconsistent retransmissions in lieu of the original
data, so that all retransmissions contained new data with the same sequence
numbers. Listen ports were set as 80, 4001, and 5001. Port 80 is used by web
traffic, and can be expected to be affected by middleboxes such as
``transparent'' caches and firewalls. We expect ports 4001 and 5001 to be less
likely to be subject to interference by middleboxes, since they are not used by
popular applications.

% Port 5001 is used for control, i.e. with no induced loss.

% Clients were deployed across a number of access networks, operated by different
% service providers. Each
Clients connected to the server, and received data. All
incoming segments to the client host were recorded by \texttt{tcpdump}, then
filtered by \texttt{iptables} to uniformly drop 5\% of segments before reaching
the TCP stack for traffic from ports 80 and 4001. Traffic from port 5001 was
unaffected.\footnote{Given that our goal is to test the ability to deploy TCP
Hollywood, rather than performance, we are only concerned with creating
sufficient loss to trigger inconsistent retransmissions. A high
\emph{un-correlated} drop rate enables TCP to survive where it would fail
against correlated drops. The ensuing reduction in throughput translates to
reduced loss due to congestion. Thus the client is more likely to see both the
original transmission and its retransmission.} Each loss induced at the client
triggered an inconsistent retransmission from the server.
%The random drop ensured that the client TCP stack experienced loss, and
%triggered the retransmission of the lost segment from the server.
Remaining segments were passed up the stack to the client application, as
normal. Data received by the client application was recorded, and compared
against \texttt{tcpdump} logs from the server to identify the dropped
segments, and to compare the payload data in the dropped segments with
that sent in the original packet and in the inconsistent retransmission.
This allows us to see what segments have been dropped, and to confirm that
both the original and retransmission cross the path between client and
server, and whether the inconsistent retransmission was delivered.

The evaluation was conducted using clients in 14 different locations in the UK,
connecting to a server located at the University of Glasgow. The clients
connected via eight different fixed-line residential ISPs (Andrews \& Arnold,
BT, Demon, EE, Eclipse, Sky, TalkTalk, and Virgin), and four mobile operators
(EE, O2, Three, and Vodafone).  All of the fixed-line residential ISPs
successfully delivered the inconsistent retransmissions. Among the four mobile
operators, only one delivered inconsistent retransmissions. The three remaining
mobile operators delivered the original segments instead, yet the server saw no
corresponding segment loss. This observed behaviour is consistent with a
transparent split-connection TCP performance enhancing proxy cache that
intercepts and responds to ACKs from the client on behalf of the server. This
caching behaviour was observed on ports 80 and 4001 for two of the three
providers, while the other provider appeared to operate a cache on port 80 only.

Crucially, TCP Hollywood {\em continued to operate irrespective of middlebox
presence} in the network. At no time did connections suffer a reset, nor did the
use of the TCP Hollywood extensions affect connectivity or performance.
Middlebox manipulations such as caching are designed to be transparent, leaving
the client to believe it is interacting with a standard TCP server. Recall from
Section~\ref{subsec:partialdep} that TCP Hollywood is designed for partial
deployment. This experiment provides evidence that TCP Hollywood continues to
deliver messages and eliminate head-of-line blocking, even when inconsistent
retransmissions are absent. In the worst-case, performance is the same as TCP
without our extensions.

The set of networks tested is by no means exhaustive. Further, and larger scale,
evaluation is needed to build evidence that inconsistent retransmissions are
deployable. Previous studies provide room for optimism, however. Honda et al.
\cite{honda:2011:extend-tcp} investigated deployment of TCP modifications with
regards to middlebox interaction, from 142 networks in 24 countries, in early
2011, including inconsistent retransmission measurements taken over a large
number of paths, with path diversity. Their observations mirror ours: the
majority of paths deliver inconsistent retransmissions as expected, while a
small number deliver the original instead. They also observed connection resets
on one path, representing less than 1\% of paths evaluated.

%Some discussion about (hypothetical, unseen)
%middleboxes that might break the protocol? And checksumming/DTLS as a solution}
