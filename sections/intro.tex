% !TEX root = ../utltcp-paper.tex

Real-time networked multimedia applications have long contributed to
Internet traffic. This can take the form of telephony \cite{RFC3261},
video conferencing \cite{jennings:2013:webrtc}, live or on-demand TV
and movies \cite{cha:2008:watching-tv,stockhammer:2011:dash}, or
user-generated video. These applications, and the traffic they generate,
are rapidly increasing in popularity, and now
comprise the majority of Internet traffic \cite{cisco:2013:vni-forecast}.

The nature of such real-time traffic is that it prefers predictable and bounded
latency to strict reliability, since data that arrives too late is as
bad as data that does not arrive at all. This suggests that data should
be sent in packets that can be independently decoded
\cite{clark:1990:architecture}, to allow them to be processed irrespective
of the loss or delay of other packets. However, the requirement for efficient media
compression leads to interdependence between packet contents and codecs
that operate across multiple frames.  When coupled with
challenging network environments, such as mobile wireless, that have
unreliable delivery and unpredictable latency, the requirements for
effective media transport become difficult to satisfy.

Applications access the network via the transport layer. The transport
protocol should provide services to meet the application demands,
abstracting away details of the transport process, and delivering data with an
appropriate degree of reliability and timeliness. For real-time networked
multimedia, the transport should be trusted to minimize transport-induced
delay, and should respect (partial) reliability semantics pertaining to
media importance, deadlines, and dependencies.

Message-oriented transports, such as SCTP \cite{rfc:4960} and DCCP
\cite{rfc:4340}, ought to be suitable building blocks, but their deployment
is restricted by NATs, firewalls, and other middleboxes
\cite{hatonen:2010:gateway}. This leaves real-time applications to use
UDP or TCP, neither of which is well-suited to their needs. UDP
contributes minimal latency, making it the recommended transport to meet the
strict latency bounds of real-time applications
\cite{draft-ietf-rtcweb-rtp-usage}, but provides limited support to
applications, and is commonly blocked by enterprise firewalls. TCP prefers
reliability to timeliness, and its congestion control tends to drive up
queueing delay, but is often the only transport that can pass through
middleboxes on the path. Accordingly, and despite its problems,
TCP is rapidly becoming the de facto transport for multimedia traffic.

In this paper, we engineer TCP Hollywood in response to these trends.
TCP Hollywood is an unordered and time-lined transport protocol, that is wire
compatible with standard TCP, but eliminates two sources of transport-induced
latency, and provides reliability semantics that better suit real-time
multimedia applications.  Specifically, TCP Hollywood: 1) removes
head-of-line blocking at the receiver and delivers received data to the
application immediately, irrespective of ordering; and 2) relaxes
reliability to respect time lines provided by the application, so only
data that will arrive in time is retransmitted, otherwise retransmissions
carry new data. The combination of both design elements reduces latency
and introduces message-oriented semantics, allowing TCP Hollywood to
express inter-dependencies between messages. Crucially, TCP Hollywood is
wire-compatible with TCP, and incrementally deployable on the public
Internet.

Our implementation consists of an intermediate logic layer that sits
between the application and the kernel. Extensions in the TCP stack
facilitate out-of-order delivery, and can be enabled or disabled via
socket options. Messages are delineated in the logic layer using timing
and dependency information from the application, and COBS-encoded
\cite{chesire:1997:COBS} to survive re-segmentation that may occur in the
network. We introduce the concept of inconsistent retransmissions: if the
round-trip time (RTT) estimator indicates that a message will arrive too late to be useful,
or if a message depends on a previous unsuccessfully transmitted message,
then TCP Hollywood can exploit re-transmission slots to send new data and
avoid retransmitting useless data. The semantics of TCP are maintained by
preserving the sequence numbers in retransmitted segments, whether
inconsistent or not.
We develop an analytical framework to model the value of a retransmission
against the buffering and processing time of data at the receiver-side.
Our analysis reveals a wide range of RTT values where standard TCP
retransmissions will arrive too late to be useful. We use this model to
validate TCP Hollywood, and show that it handles retransmissions correctly.

Our contributions are as follows.  After reviewing the rationale and
requirements in Section \ref{sec:background}, we design a TCP-compatible
architecture and application programming interface (API) that eliminates
transport related, but not congestion control related, delay from TCP in
Section \ref{sec:design} (this can be used with any of the existing
proposals to reduce congestion control related delay, such as active
queue management \cite{nichols:2012:codel,khademi:2014:new-aqm} or
delay-based congestion control \cite{jin:2004:fast-tcp,brakmo:1994:tcp-vegas}).
We develop an analytic framework in Section
\ref{sec:analysis} to determine the value and content of retransmitted
data. Section \ref{sec:perfeval} describes performance evaluations based on 
a simulated VoIP application, with typical cross-traffic classes. 
Experiments to demonstrate the ease of deployment of TCP Hollywood are
discussed in Section \ref{sec:deployability}. Related work and concluding
remarks are provided in Sections \ref{sec:related} and
\ref{sec:conclusions}, respectively.

This paper is an extended version of \cite{mcquistin:2016:tcp-hollywood}.
It revises and extends the material in that paper, extending the discussion 
of service requirements and APIs for real-time applications in Section
\ref{sec:background}, adding a performance evaluation in Section
\ref{sec:perfeval}, and adding an appendix discussing reproducibility of
our results.

%==================================================================================================
