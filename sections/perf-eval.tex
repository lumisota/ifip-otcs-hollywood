% !TEX root = ../utltcp-paper.tex

\narrative{The analysis introduced three applications: telephony, 
streaming video, and live video. It showed when our modifications are
active. This section performs two types of performance evaluation, in the
real-world, and on a testbed. The real-world tests are designed to show
that our modifications are triggered in real networks, and the performance
benefits gained. The testbed gives us a controlled environment, in which
we can test scenarios not found by the (limited) real-world tests.}

Introduction to section: builds on analysis, showing performance gains.
Outline real-world vs testbed - why both?

%--------------------------------------------------------------------------------------------------
\subsection{Real-World}
\label{sec:perf:realworld}

Intro: hosts in residential network near Glasgow, UK, and in datacenters in
London, UK, and San Francisco, USA. Locations selected to test two ranges
of RTTs: intra and intercontinental.

We've chosen to evaluate two applications: voice telephony, and live video
streaming. The analysis shows that these two applications should benefit
from TCP Hollywood.

We'll measure the following metrics:

\begin{description}
  \item[Timely goodput] \hfill \\
  This is the number of \textit{usable} bytes delivered to the application.
  A byte is usable if it arrives on time to be played out. We measure this
  for both TCP Hollywood (either full or partial deployment), and for TCP
  at both hosts.
  \item[Number of inconsistent bytes in retransmissions] \hfill \\
  This is useful to show where goodput is increased by the sender-side
  modifications.
  \item[Number of bytes delivered out-of-order] \hfill \\
  This is useful to show where goodput is increased by the receiver-side
  modifications.
\end{description}

For all three of these metrics, we will give a total for the entire session,
and graph the value over one second intervals. The totals are useful as an
overview, while the graphs will be better at showing how the modifications
impact timely goodput.

We present three sets of real-world evaluations, testing full deployments
for both voice telephony and live video streaming, and a sender-only
deployment of voice telephony. A sender-only configuration is tested as
content providers will be able to modify their sending stacks, but are
unlikely to be able to modify receivers.

\subsubsection{Voice telephony}

For this application, we simulate a G.711 voice telephony application.
160 byte messages will be sent every 20ms, with simulated streams lasting
120 seconds.

Both sender and receiver have TCP Hollywood kernel modifications deployed.

Play-out delay is fixed, and based on jitter measurements carried out
before the simulations (and so there are two play-out delays, one for
Glasgow<->London, and for Glasgow<->San Francisco). The play-out delay
will be calculated such that 99\% of packets arrive on time to be used.

\begin{figure*}[t]
	\subfloat[Timely goodput\label{fig:voip:lon:gput}]{%
		\rule{0.3\textwidth}{4cm}
	}
	\hfill
	\subfloat[Inconsistent bytes\label{fig:voip:lon:incon}]{%
		\rule{0.3\textwidth}{4cm}
	}
	\hfill
	\subfloat[Out-of-order bytes\label{fig:voip:lon:ood}]{%
		\rule{0.3\textwidth}{4cm}
	}
	\centering
	\caption{Voice telephony (G.711) simulation from London, UK to Glasgow, UK - both TCP Hollywood hosts}
	\label{fig:voip:lon}
\end{figure*}

Figure \ref{fig:voip:lon} shows the three metrics, as measured when
simulating the application sending from London, UK, to Glasgow, UK.

\begin{figure*}[t]
	\centering
	\subfloat[Timely goodput\label{fig:voip:sf:gput}]{\rule{0.3\textwidth}{4cm}}
	\hfill
	\subfloat[Inconsistent bytes\label{fig:voip:sf:incon}]{\rule{0.3\textwidth}{4cm}}
	\hfill
	\subfloat[Out-of-order bytes\label{fig:voip:sf:ood}]{\rule{0.3\textwidth}{4cm}}
	\caption{Voice telephony (G.711) from San Francisco, UK to Glasgow, UK - both TCP Hollywood hosts}
	\label{fig:voip:sf}
\end{figure*}

Figure \ref{fig:voip:sf} shows the three metrics, as measured when
simulating the application sending from San Francisco, USA, to Glasgow, UK.

\subsubsection{Live video streaming}

Application based on the low-latency DASH application outlined in analysis.

\begin{figure*}[t]
	\centering
	\subfloat[Timely goodput\label{fig:video:lon:gput}]{\rule{0.3\textwidth}{4cm}}
	\hfill
	\subfloat[Inconsistent bytes\label{fig:video:lon:incon}]{\rule{0.3\textwidth}{4cm}}
	\hfill
	\subfloat[Out-of-order bytes\label{fig:video:lon:ood}]{\rule{0.3\textwidth}{4cm}}
	\caption{Video streaming simulation from London, UK to Glasgow, UK - both TCP Hollywood hosts}
	\label{fig:voip:lon}
\end{figure*}

Figure \ref{fig:voip:lon} shows the three metrics, as measured when
simulating the application sending from London, UK, to Glasgow, UK.

\begin{figure*}[t]
	\centering
	\subfloat[Timely goodput\label{fig:video:sf:gput}]{\rule{0.3\textwidth}{4cm}}
	\hfill
	\subfloat[Inconsistent bytes\label{fig:video:sf:incon}]{\rule{0.3\textwidth}{4cm}}
	\hfill
	\subfloat[Out-of-order bytes\label{fig:video:sf:ood}]{\rule{0.3\textwidth}{4cm}}
	\caption{Video streaming from San Francisco, UK to Glasgow, UK - both TCP Hollywood hosts}
	\label{fig:video:sf}
\end{figure*}

Figure \ref{fig:video:sf} shows the three metrics, as measured when
simulating the application sending from San Francisco, USA, to Glasgow, UK.

\subsubsection{Sender-only deployment}

This is where only the sender has TCP Hollywood modifications deployed.
An important configuration to test: it is likely to be the most common.
Content providers are likely to be able to modify the sender stack, but not
the receiver-side.

As there are no receiver-side modifications, we drop the out-of-order bytes
metric -- it doesn't apply here.

The evaluation is performed using the same simulated voice telephony
application used above.

Without the receiver-side modifications, head-of-line blocking remains.
As shown in Figure \ref{diagram:motivation_latency}, head-of-line blocking
accounts for a larger proportion of the latency TCP introduces, compared
with retransmissions. The figure shows four packets delayed, three of which
are delayed by head-of-line blocking. Future work should include a
mechanism that allows the receiver to notify the sender that the receiver
side modifications are not in use. Then, in the event of a retransmission,
the sender is aware that a certain number of messages are being delayed.

\begin{figure}[t]
	\centering
	\subfloat[Timely goodput\label{fig:voip:sender:lon:gput}]{\rule{\columnwidth}{4cm}}
  \\
	\subfloat[Inconsistent bytes\label{fig:voip:sender:lon:incon}]{\rule{\columnwidth}{4cm}}
	\caption{Voice telephony (G.711) simulation from London, UK to Glasgow, UK -- TCP Hollywood on sender only}
	\label{fig:voip:sender:lon}
\end{figure}

Figure \ref{fig:voip:sender:lon} shows the two metrics, as measured when
simulating the application sending from London, UK, to Glasgow, UK.

\begin{figure}[t]
	\centering
	\subfloat[Timely goodput\label{fig:voip:sender:sf:gput}]{%
		\rule{0.45\textwidth}{4cm}
	}
	\\
	\subfloat[Inconsistent bytes\label{fig:voip:sender:sf:incon}]{%
		\rule{0.45\textwidth}{4cm}
	}
	\caption{Voice telephony (G.711) from San Francisco, UK to Glasgow, UK -- TCP Hollywood on sender only}
	\label{fig:voip:sender:sf}
\end{figure}

Figure \ref{fig:voip:sender:sf} shows the two metrics, as measured when
simulating the application sending from San Francisco, USA, to Glasgow, UK.

%--------------------------------------------------------------------------------------------------
\subsection{Testbed}
\label{sec:perf:testbed}

Using a testbed gives us control and flexibility vs the real-world
evaluations. We'll test a range of RTT/play-out delay combinations near
the regular TCP retransmission boundary.
