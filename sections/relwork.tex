% !TEX root = ../utltcp-paper.tex

The immediate precursors of TCP Hollywood are the Minion protocol suite
\cite{nowlan:2012:minion} and TL-TCP \cite{mukherjee:2000:timelines}. The
Minion protocol suite includes uTCP, which proves a COBS-encoded user-space
datagram abstraction atop TCP, with prioritization and out-of-order
delivery. uTCP also provides an API that enables applications to replace
existing datagrams in the transmission buffer before they are sent.
Datagrams that have already been sent (i.e., those being retransmitted)
cannot be replaced. The authors acknowledge this as a conservative design
choice, made to ensure middlebox interaction.

Our wire compatibility experiments from Section \ref{sec:deployability},
and those of Honda et al. \cite{honda:2011:extend-tcp}, indicate that
inconsistent retransmissions are possible, but that the integrity of the
sequence space needs to be preserved. The need to consider middlebox
interaction with new or modified protocols is underscored by the design,
and success, of Multi-Path TCP \cite{raiciu:2012:hard}. The design of
TCP Hollywood builds on a number of protocols, and tweaks to TCP, that are
unlikely to be deployable.

TL-TCP marks the first appearance of time-lines and
inconsistent retransmissions~\cite{mukherjee:2000:timelines}. The underlying
mechanism works by injecting gaps into the sequence space. This modification is
observable by middleboxes, and so is unlikely to be deployable. TCP Hollywood
builds on TL-TCP, and related protocols, but does so while focussing on
deployability. As discussed in Section \ref{sec:design}, we minimise changes
to the wire protocol to maximise
compatibility with middleboxes.

Transport protocols that rely on application-layer metadata to
improve performance include Partially Error Controlled Connection
(PECC)~\cite{dempsey:1992:adaptive} and PRTP-ECN~\cite{grinnemo:2001:prtp}.
Other protocols such as SCTP \cite{rfc:4960} and DCCP \cite{rfc:4340} were
engineered to broaden the delivery models offered by the transport-layer.
Despite standardization and deployment in mainstream operating systems, their
use is hampered by a lack of middlebox support.
% The main difference between these protocols and TCP Hollywood is the focus on
% deployability: middlebox interaction must be considered.

Liang and Cheriton in~\cite{liang:2002:tcp-rtm} note that loss can be more detrimental
to streaming application performance than jitter. On-demand streaming
applications, for example, can effectively hide jitter from the application but
are unable to tolerate loss. The authors present a modified TCP, TCP-RTM, that
allows receivers to read beyond a gap in the receive buffer. The sequence
numbers in the gap are ACKed, preventing their retransmission by the sender.
Applications read from the socket at a predetermined play-out rate offset by
some delay. There are no changes to TCP itself; instead, the interaction
between application and receiver buffer is modified. Selective negative
ACKs (NACKs) allow senders to be informed of the segments that were skipped
over.

Deadline-aware TCP is a modified TCP  specifically for datacenters, and
implements flows with soft time constraints~\cite{vamanan:2012:d2tcp}. The
modifications allow for the TCP window size and congestion back-off to be varied
based on the flow congestion deadline. Flows with imminent deadlines benefit
from larger windows. As the network becomes congested, flows will tend to
complete closer to their deadlines. The modifications require ECN support in the
network, and a modified TCP sender. Requiring ECN support effectively prevents
deployment outside of datacenters.

QUIC (Quick UDP Internet Connections) \cite{IS15} is a transport-layer protocol
implemented atop UDP. It incorporates a number of latency-reducing techniques
(e.g., large initial data transfers, low RTT setups) that are slowly migrating
to TCP. Its use of UDP as a substrate provides an interesting contrast to our
choice of TCP, but follows from the architectural principles described in Section
\ref{sec:background-tcp} (i.e., that TCP and UDP are substrates for novel transport
layer protocols). Implementing atop UDP allows QUIC to operate entirely in userspace,
greatly improving the likelihood of deployment. However, the flexibility
of a userspace implementation comes at the cost of universal deployment. Upon detection 
of a blocking device QUIC is forced to
fall back to TCP; QUIC authors indicate that this happens in around 5-10\% of cases, based
on initial results. We therefore view TCP Hollywood and QUIC as complementary, rather
than competing, protocols: falling back from QUIC to TCP Hollywood, rather than standard
TCP, is likely to offer performance benefits to latency-sensitive applications.

The trade-off between a UDP-based protocol with fall-back to standard TCP,
as chosen by the QUIC authors, and a slightly modified TCP variant, as we
have chosen, hinges on ease of implementation and deployment. We believe
our implementation is simpler, since we build on the TCP infrastructure,
but acknowledge that this gives us less flexibility to evolve the protocol.
Equally, we believe our implementation is likely to be more deployable, as
it builds on TCP. Broader measurement studies, for both TCP Hollywood and
QUIC, are needed to evaluate this claim, however.
